\documentclass[10pt,a4paper]{article} % 参考 https://www.kancloud.cn/digest/latexeverywhere/181887

\usepackage{titlesec} % \titleformat

\usepackage{indentfirst}             % 段首缩进

% 字体相关
\usepackage{fontspec}
\setsansfont{Times New Roman}        % Calibri
\XeTeXlinebreaklocale "zh"           % 中文断行
\usepackage[slantfont,boldfont]{xeCJK}
\setCJKmainfont[BoldFont=SimHei]{FangSong}   % 设置缺省中文字体 % FangSong=仿宋,SimSun=宋体,STKaiti=楷体,Microsoft YaHei=微软雅黑

% Define light and dark Microsoft blue colours
\usepackage{xcolor}
\definecolor{MSBlue}{rgb}{.204,.353,.541}
\definecolor{MSLightBlue}{rgb}{.31,.506,.741}
% Define a new fontfamily for the subsubsection font
% Don't use \fontspec directly to change the font
\newfontfamily\subsubsectionfont[Color=MSLightBlue]{Times New Roman}
% Set formats for each heading level
\titleformat*{\section}{\Large\bfseries\sffamily\color{MSBlue}}
\titleformat*{\subsection}{\large\bfseries\sffamily\color{MSLightBlue}\song}
\titleformat*{\subsubsection}{\itshape\subsubsectionfont}

% Header and Footer
\usepackage{fancyhdr}
\pagestyle{fancy}
\fancyhead{}
\fancyfoot{}
\fancyfoot[R]{ \thepage\ }
\renewcommand{\headrulewidth}{0pt}
\renewcommand{\footrulewidth}{0pt}

% 文献引用
\usepackage{cite}
\renewcommand{\refname}{参考文献}

% 页边距调整
\usepackage[margin=1in,left=1in,includefoot]{geometry}
%\usepackage[left=2cm,right=2cm,top=2cm,bottom=2cm]{geometry}

% 超链接
\usepackage[hidelinks]{hyperref}

% 数学包
\usepackage{amsmath}
\usepackage{amsfonts}
\usepackage{amssymb}
\usepackage{xfrac} % use \sfrac for slanted fraction instead of \frac

% 图片
\usepackage{graphicx}
\usepackage{float}

% TikZ画图等
\usepackage{tikz}
\usetikzlibrary{arrows}
\usepackage{circuitikz}
\usetikzlibrary{calc}

\begin{document}

%\pagestyle{empty}
%
\tikzstyle{int}=[draw, fill=blue!20, minimum size=2em]
\tikzstyle{init} = [pin edge={to-,thin,black}]

\begin{titlepage}
    \begin{center}
        \line(1,0){440}\\
        [0.25in]
        \huge{\bfseries 永磁同步电机参数自整定与无编码器运行功能}\\
        [2mm]
        \line(1,0){400}\\
        [1.5cm]
        \textsc{\LARGE \bfseries 杭州}\\
        [10cm]
    \end{center}
    \begin{flushright}
    \textsc{\Large 陈嘉豪\\2020~年~2~月}
    \end{flushright}
\end{titlepage}

% Front Matter
\pagenumbering{roman}
\section*{致谢}
\addcontentsline{toc}{section}{\numberline{}致谢}
\cleardoublepage

% Table of Contents
\renewcommand{\contentsname}{目录}
\tableofcontents
\thispagestyle{empty}
\cleardoublepage
\setcounter{page}{1}

% Main Body
\pagenumbering{arabic}
\section{引言}\label{sec:intro}
我是中文字\cite{2019-Chen.Zhu.ea-Review}。

{\bfseries 我是加粗后的中文字。}

There is a theory which states that if ever anyone discovers exactly what the Universe is for and why it is here, it will instantly disappear and be replaced by something even more bizarre and inexplicable.
There is another theory which states that this has already happened.

% \begin{figure}[h!]
% \centering
% \includegraphics[scale=1.7]{universe}
% \caption{The Universe}
% \label{fig:universe}
% \end{figure}

\begin{table}[H]
    \centering
    \begin{tabular}{c|c}
         &  \\
         &
    \end{tabular}
    \caption{Caption}
    \label{tab:motor_performance}
\end{table}

\begin{table}[!t]
  \caption{List of Symbols}    % title of Table
  \centering                % used for centering the whole table
    \begin{tabular}{cccc}
        % after \\: \hline or \cline{col1-col2} \cline{col3-col4} ...
        \hline
        \hline
        \emph{Inverse-$\Gamma$-circuit Symbols} & \emph{Notation} & \emph{Comment} \\
        \hline
        Stator resistance   & $r_s$             & $3.04~\Omega$    \\
        Output error of high gain AO & $\tilde y$ & $\tilde y= i_s - \hat i_s$ \\
        High gain coefficient        & $\vartheta$ & - \\
        \hline
    \end{tabular}
  \label{tab:001}      % is used to refer this table in the text
\end{table}

\section{结论}
``I always thought something was fundamentally wrong with the universe'' \cite{adams1995hitchhiker}



\begin{tikzpicture}[node distance=2.5cm,auto,>=latex']
    \node [int, pin={[init]above:Load}] (a) {Induction Motor};
    \node (b) [left of=a,node distance=4cm, coordinate] {};
    \node (c) [right of=a,node distance=4.5cm] {};
%    \node [coordinate] (end) [right of=c, node distance=2cm]{};
    \path[->] (b) edge node {Stator current} (a);
    \path[->] (a) edge node {Torque on shaft} (c);
%    \draw[->] (c) edge node {$p$} (end) ;
\end{tikzpicture}

\begin{tikzpicture}[node distance=2.5cm,auto,>=latex']
    \node [int, pin={[init]above:Load, Gravity}] (a) {Bearingless Induction Motor};
    \node (b) [left of=a,node distance=6.5cm, coordinate] {};
    \node (c) [right of=a,node distance=5.5cm] {};
%    \node [coordinate] (end) [right of=c, node distance=2cm]{};
    \path[->] (b) edge node[align=center]{Sator torque current\\ Stator suspension current} (a);
    \path[->] (a) edge node[align=center]{Torque on shaft\\ Force on rotor} (c);
%    \draw[->] (c) edge node {$p$} (end) ;
\end{tikzpicture}


\begin{circuitikz}
  %% Source
  \coordinate (Csource);
  \foreach \anch/\ang in {a/0,b/120,c/240}{%
    \draw (Csource) to[vsourcesin=$\angle \ang^\circ$,*-*] +({-\ang+90}:3) coordinate (S\anch) node[anchor={\ang}]{(S\anch)};
  }
  %% Load
  \coordinate (Cload)  at ($(Csource)+(10,0)$);
  \foreach \anch/\ang in {a/0,b/-120,c/-240}{%
    \draw (Cload) to[R=$R_{\anch}$,*-*] +({-\ang+90}:3) coordinate (L\anch) node[anchor={\ang}]{(L\anch)};
  }
  %% Connections
  \draw (Sa) -- +(0,1) to[R] ($(La)+(0,1)$) -- (La);
  \draw (Sb) to[R] (Lb);
  \draw (Sc) -- +(0,-2) to[R] ($(Lc)+(0,-2)$) -- (Lc);
\end{circuitikz}


\bibliographystyle{IEEEtran}
%\bibliography{references}

\end{document}


